\documentclass[10pt,a4paper]{article}
\usepackage[utf8]{inputenc}
\usepackage{amsmath}
\usepackage{amsfonts}
\usepackage{amssymb}
\begin{document}


\title{Karkas Veritabanı Standartları}


\section{Karkas SQL Server Veritabanı Standartları}


SQL Server schema kavramını en iyi şekilde uygulayan veri tabanları arasındadır.
Bu yüzden schema yapısının SQL server veritabanı tasarımı yapılırken kullanılması tavsiye edilir.
Bu yapıda her ana modul için bir ana şema bir tanım tablosu şeması açılması tavsiye edilmektedir.
Örneğin ihtiyaç duyduğunuz modullerin Insan Kaynakları, Muhasebe ve Bordro olduğunu düşünelim. 
Bu yapıda aşağıdaki şemaların açılması uygun olacaktır.
\begin{itemize}
\item MUHASEBE
\item TT\_MUHASEBE
\item INSAN\_KAYNAKLARI
\item TT\_INSAN\_KAYNAKLARI
\item BORDRO
\item TT\_BORDRO
\item ORTAK
\item TT\_ORTAK
\item OZLUK\_BILGILERI
\item TT\_OZLUK\_BILGILERI
\item FIRMALAR
\item TT\_FIRMALAR

\end{itemize}


Burada muhasebe, insan kaynakları ve bordro şemaları zaten ihtiyaç duyulan şemalardır.
Bunlara ek olarak kişi bilgilerinin tutulması için ad, soyad, kimlik bilgileri etc ORTAK şeması
önerilmektedir. TT\_ORTAK şeması buradaki bilgiler için gerekli tanım tablolarını tutacaktır.
Örneğin TT\_ORTAK.CINSIYET ismindeki bir tanım tablosu cinsiyet bilgileri tutabilir.
Burada Kisi için tutulacak bilgiler herkes için kesin tutulması gereken bilgilerdir. 
Örneğin ad soyad, tc kimlik no resim gibi. Ama eger sadece yazılımı kullanan kurumdaki,
kişilere ait  (örneğin kurum sicil no, kurum eposta, kurum telefon) gibi bilgiler  var ise 
bunların farklı şemada tutulması daha yararlı olacaktır. 

Buraya FIRMALAR şeması muhasebe firmalarının fatura kesecekleri firma bilgileri için eklenmiştir.
Firma, Kisi bilgileri bir çok modul tarafından ihtiyaç duyulan bilgiler oldukları için bunların
ayrı şemalarda tutulmaları daha yararlı olacaktır.

Son olarak Ozluk bilgileri, Bordro ve insan kaynaklarının ihtiyaç duydukları ortak bilgileri taşıyacaktır.
Burada bu bilgiler sadece insan kaynakları şeması yada bordro şeması yerine ikisinin ortak kullandığı
bir şemada olması daha uygundur.







  			Veritabanındaki Tablo isimleri tüm harfler büyük, Kelimeler arasında 
  ise \_ olacak şekilde seçilmelidir.
  			INSAN\_KAYNAKLARI.ILAC\_ALIMI\_ALINAN\_ILACLAR		 
  		Her Modul kendisine ait bir schema içinde tanımlanmalıdır. Referential 
  Integrity korunması açısından bütün uygulama 		tek bir veritabanında 
  çalışmalıdır.		 
  			Schema ismi olarak kullandığınız isimleri tablo isimleri olarak 
  kullanmayın. Yani eğer MAAS diye bir schema'nız varsa,			bu schema icinde veya 
  diğer schemaların içinde MAAS diye bir tablonuz olmamalıdır.		 
  			VS.NET içinde proje ismi olarak kullanacağınız bir ismi Schema ismi 
  olarak kullanmayınız. Yada Schema isimlerinizi, VS.NET içinde			proje ismi 
  olarak vermeyin. Compiler için sorunlar çıkarabiliyor.		 
  			Her modul relationları, ilişkileri gösteren diagramlara sahip 
  olmalıdır. 		 
  			Raporlama açısından bütün tablo'ların ve anlamı açık olmayan kolanların 
  descriptionlarının yazılması gerekmektedir.		 
  			Tablolarda mümkün olduğunca girilen bilgiler not null yapılmalıdır.
  		 
  			Tablolarda primary key ismini verirken, TabloIsmi + Key 
  kullanabilirsiniz. KISI tablosunun primary Key'i KisiKey olabilir.		 
  			ORTAK isminde bir schema ortak kullanım için açılmalı, Bütün modüllerin 
  kullanabileceği tablolar burada oluşturulmalıdır.			KISI, KISI\_NUFUS\_BILGILERI 
  .... Bu şemada aslında nerdeyse sadece kişi bilgileri tutulmaktadır.
  Buradaki tablo sayısı sınırlı tutulmalıdır. 10 tanenin üstüne çıktığı zaman
  veritabanı daha zor anlaşılır olmaktadır.
  
  			TT\_ORTAK  (Tanım Tabloları Ortak) isminde bir schema ortak kullanım 
  için açılmalı, Bütün tanım bilgileri burda şeçilmelidir. 			Sehir, Cinsiyet 
  ...		 
  			Her modülün kendine ait bir Tanım Tabloları scheması açılmalıdır. 
  Mesela TT\_STAJ. Bu module ait tanım bilgileri burda girilmelidir. 		 
   eğer Tanım tablosundaki değer 0-255  tinyint veri yapısını kullanın.
  		 
   eğer Tanım tablosundaki değer  -32,768 - 32,767   smallint veri yapısını 
  kullanın.		 
   eğer tamsayı değer  -2,147,483,648 - 2,147,483,647 arası ise int veri 
  yapısını kullanın.		 
  
  
  			Eğer bir tablo birden fazla modul tarafından kullanılıyor ise, 
  bu ortaklığı belirtecek yeni bir şema gerekirse açılmalıdır. 
  Bu şema ismi eğer dışarıdan veri alınıyorsa bunu anlatan bir isim,
  eğer 2 modulün ortak noktası ise, bu ortak nokta olabilir.
  Sadece 2 modul tarafından kullanılacak bir tablo ORTAK şemasını eklenmemelidir.
  
  
  
  
  			Eğer anlamlı bir default/varsayılan değer verilebiliyorsa verilmelidir.
    Örneğin GuncellemeTarih default now() verilebilir.
  
  			Birincil anahtar (Primary Key) konulmayan tablo olmayacaktır.		 
  			Sentetik Anahtar (Surrogate Key) olarak int IDENTITY değerleri yerine 
  uniqueidentifier tercih edilmelidir.		 
  			Uniqueidentifier eğer PK olarak kullanılıyorsa default değer olarak 
  newid() olmalıdır.		 
  			 Money SQL standartında olmadığından kullanılmamalıdır. Ayrıca float ve 
  double tam değerler değil 			 yaklaşık değer olarak çalışırlar. Para 
  hesaplarında kesinlikle kullanılmamaları gerekir.		 
  			Para değerleri numeric(18,2) veya decimal(18,2) olmalıdır.	
  numeric ve decimal bir birlerinin alias'ıdır. 
    Eğer küsürat hesaplamalarıda gerekli ise numeric(18,4) veya daha fazla
    bir ondalık kullanılabilir.

  
  			Turkish\_CI\_AS Sunucu coallationu olarak seçilmelidir. Nchar unicode 
  olarak değerleri tuttuğu için 	2 katı yer kaplamakta ve 2 katı yavaş 
  çalışmaktadır. Eğer veritabanında aynı anda iki dil bilgisi tutulmayacaksa, 
  yani			hem rusça hemde türkçe bilgi,  varchar yada char olarak tutulmalıdır.
  		 
 Eğer bilgi girilen tip daima aynı sayıda ise örnek: Posta Kodu gibi 
  char olarak tanımlanmalıdır.		 
  			Eğer bilgi girilen değer değişken bir değer ise, ad soyad gibi varchar 
  olarak tanımlanmalıdır.		 
  			Tarih bilgilerini tutarken smalldatetime tercih ediniz. 			
  Eğer sadece 
  yıl bilgisi tutacaksanız, Mezuniyet yılı smallint tutabilirsiniz. Bu tür bir 
  kolona constraint ekleyin.			örn: 
  
  $  CHECK  (([GirisSene]>=(1900) AND   [GirisSene]<=datepart(year,getdate()))) $
  
  			 <BR>			Yukarıdakı kısıtlama 
  giris senesinin 1900 ile şu an bulunduğumuz sene arasında kalmasını 
  sağlamaktadır.								 
  				datetime (3.33 milliseconds accuracy)  	Jan 1, 1753  	Dec 31, 9999
  				 <BR>				smalldatetime (1 minute accuracy) 	Jan 1, 1900 	Jun 6, 2079		
  				 <BR>değerleri arasını göstermektedir. .NET üstünde doğru kabul edilen, 
  örn: 01.01.1600 gibi bir değer SQL Server'da				hata verebilir. Tarih 
  Değerlerinin bu aralıkta olduğuna emin olunuz.						 
  			Resim veya binary data tutmak için varbinary(max) tercih edin. image 
  veri tipi SQL Server'ın yeni versiyonlarında			kullanılmayacaktır.		 
  			Eğer bir tablo içindeki kolon başka bir tabloya referans veriyorsa 
  kesinlikle foreign key tanımlanmalı ve bu			kolonun ismi TabloIsmiTipNo yada 
  TabloIsmiKey olarak tanımlanmalı. TipNo tanım tabloları için. Key ise normal 
  tablolar			için kullanılmalıdır.	
    TipNo yerine referans verilen tanımın türkçesine göre TurNo
    gibi diğer kelimelerde kullanılabilir.
    Eğer bir kolon ismi TurNo,TipNo,Turu,Tipi,RenkNo,BilgiNo
    gibi bir kelime ile bitiyorsa ise çok büyük ihtimal ile
    Tanım Tablolarına referans veriyordur.
    

  
  			TipNo(...) okunduğu zaman bir tanım tablosuna referans düşünülürken, Key ise 
  1..* veya 1..1 ilişkisini tanımlamalıdır.		 
  			Yazılım geliştiriciler için sınırlı yetkili veri tabanı kullanıcıları 
  açılacaktır. SQL Sever Management Studio üzerinde işlem			yapılırken bu 
  kullanıcı ile sisteme girilecektir.		 
  			Yazılım geliştiriciler Stored Procedure yazma gereksinimi duyarlarsa 
  bunu veritabanı yöneticisi ve takım liderlerinin onayını			 aldıktan sonra 
  yapabileceklerdir.		 
  			Stored Procudures usp\_ ile, functionlar ufn\_ ile, triggerler ise 
  utnX\_ ile başlamalıdır (X-->;i:insert, u:update, d:delete). 			
  
  
  Note: eğer bir 
  stored Procedure sp veya xp ile başlarsa SQL server 			bunun System stored 
  procedure'u olduğunu düşündüğü için ilk önce master veritabanında arayacaktır. 
  
  
  Stored Procudure'lara isim verirken tablo ismi + yaptığı işlem şeklinde 
  isim verin.
		usp\_Personel\_Ekle			usp\_Personel\_Sil			 
		usp\_Personel\_Guncelle		usp\_Personel\_Oku			 
		usp\_Personel\_Oku\_Tum		usp\_Personel\_Ara\_AdiVeSoyadiIle
  
     Table and View-Tablo ve Görüntü 
     -- 
    INSAN\_KAYNAKLARI.MUSTERI 
      			TANIM\_TABLOLARI.NUFUS\_CUZDANI\_VERILIS\_NEDENI 
    Schema ismi + Tablo ismi, 2 kelimeden fazla olan isimler \_ ile 
      ayrılıyor. Tablo ve Viewlara ayrı isimler vermiyoruz.		  
  
     Column-Kolon 
     -- 
    MusteriKey,TipNo,SonDegistirmeTarihi 
    Her kelimenin ilk harfi büyük yazılır. 
  
     Temp Table-Geçici Tablo 
     GECICI.TABLO\_ISMI 
    MusteriKey,TipNo,SonDegistirmeTarihi 
    Her kelimenin ilk harfi büyük yazılır. 
  
     Index isimlendirme 
     IX\_TabloIsmiKolonIsmi 
     IX\_MusteriMusteriNo 
    Her kelimenin ilk harfi büyük yazılır. 
  
     Yabancı anahtar (Foreign key) 
     FK\_TabloIsmi1KolonIsmi1FK\_TabloIsmi2KolonIsmi2 
     FK\_MusteriMusteriNoSiparisMusteriNo 
    Her kelimenin ilk harfi büyük yazılır. 
  
     Stored Procedure -Gömülü Yordamlar 
     usp\_ 
     usp\_PersonelAraAdiVeSoyadiIle 
     Tablo Ismi + yaptığı iş 		  
  
    Default - Varsayılan 
     DF\_ 
     DF\_PERSONEL\_IlkGirisTarihi 
     SQL Server DF veriyor. Management Studio icinde Default'ları 
      verirseniz, İsimlendirme düzgün olur. 		  
  
     Default - Varsayılan 
     CK\_ 
     CK\_OKUL\_BILGILERI\_GirisSenesi 
     SQL Server CK veriyor. Management Studio icinde Constarint 
      yazarsanız, İsimlendirme düzgün olur. 				 
  
     Trigger - Tetikleyici 
     utrX\_ 
     utru\_PERSONEL 
      utrX + Tablo Ismi, yerine triggerin hangi işlem için olduğunu 
      söyleyen bir kelime gelicektir. insert (i), update (u), delete (d)
  
\subsection{Kurulum Notları}
Veritabanı kurulurken, SQL Server içinden Collation olarak Turkish\_CI\_AS 
seçilmelidir. 
  
      
      
Revision 
Yazarlar 
Yapılan Değişiklikler 
  
     0.2  
     Atilla Özgür 
    İlk Yazım 
    
  
     0.3 (2013-02-13) 
     Atilla Özgür 
    Çok uzun bir aradan sonra bazı düzeltmeler 
    
    


\end{document}
