\documentclass[10pt,a4paper]{article}
\newcommand{\authorName}{Atilla \"Ozg\"ur}
\newcommand{\titleName}{Asp.Net MVC Scaffolding}

\usepackage{ucs}

\interdisplaylinepenalty=2500
\usepackage{algorithmic}

\usepackage{cite}



% *** MATH PACKAGES ***
%
\usepackage{amsfonts}
\usepackage{amsmath}
\usepackage{amssymb}



\usepackage{csquotes}
\usepackage{makeidx}
\usepackage{float}
\usepackage{longtable}
\usepackage[obeyDraft]{todonotes}
% \usepackage[disable]{todonotes}


\usepackage{cite}
\makeatletter
\@ifclassloaded{IEEEtran}{}{\usepackage{appendix}}
\@ifclassloaded{IEEEtran}{}{\usepackage{ltxcmds}}

\makeatother

% \usepackage{appendix}
\usepackage[final]{pdfpages}
\usepackage{import}

\usepackage{array}
% Frank Mittelbach's and David Carlisle's array.sty patches and improves
% the standard LaTeX2e array and tabular environments to provide better
% appearance and additional user controls. As the default LaTeX2e table
% generation code is lacking to the point of almost being broken with
% respect to the quality of the end results, all users are strongly
% advised to use an enhanced (at the very least that provided by array.sty)
% set of table tools. array.sty is already installed on most systems. The
% latest version and documentation can be obtained at:
% http://www.ctan.org/tex-archive/macros/latex/required/tools/


\usepackage{mdwmath}
\usepackage{mdwtab}
% Also highly recommended is Mark Wooding's extremely powerful MDW tools,
% especially mdwmath.sty and mdwtab.sty which are used to format equations
% and tables, respectively. The MDWtools set is already installed on most
% LaTeX systems. The lastest version and documentation is available at:
% http://www.ctan.org/tex-archive/macros/latex/contrib/mdwtools/


%% The lineno packages adds line numbers. Start line numbering with
%% \begin{linenumbers}, end it with \end{linenumbers}. Or switch it on
%% for the whole article with \linenumbers after \end{frontmatter}.
% \usepackage{lineno}


% \enquote{quote} şeklinde kullanılıyor.

 
 
 
\usepackage{caption}
\usepackage{subcaption}


\floatstyle{boxed} 
\restylefloat{figure}
\usepackage{multirow} 

\usepackage{hyperref}

% correct bad hyphenation here
\hyphenation{op-tical net-works semi-conduc-tor}

\usepackage{url}

\hypersetup{
    bookmarks=true,         % show bookmarks bar?
    unicode=true,          % non-Latin characters in Acrobat’s bookmarks
    pdftoolbar=true,        % show Acrobat’s toolbar?
    pdfmenubar=true,        % show Acrobat’s menu?
    pdffitwindow=false,     % window fit to page when opened
    pdfstartview={FitH},    % fits the width of the page to the window
    pdftitle={\titleName },    % title
    pdfauthor={\authorName },     % author
    pdfsubject={\titleName},   % subject of the document
    pdfcreator={\authorName},   % creator of the document
    colorlinks=false,       % false: boxed links; true: colored links
    linkcolor=red,          % color of internal links
    citecolor=green,        % color of links to bibliography
    filecolor=magenta,      % color of file links
    urlcolor=cyan           % color of external links
}



\usepackage{color}
\usepackage{courier}
\usepackage{xcolor}
\usepackage{listings}

\usepackage{caption}
\DeclareCaptionFont{white}{\color{white}}
\DeclareCaptionFormat{listing}{\colorbox{gray}{\parbox{\textwidth}{#1#2#3}}}
\captionsetup[lstlisting]{format=listing,labelfont=white,textfont=white}


\lstset{
         basicstyle=\footnotesize\ttfamily, % 
         %numbers=left,               % 
         numberstyle=\tiny,          % 
         %stepnumber=2,               % 
         numbersep=5pt,              % 
         tabsize=2,                  % 
         extendedchars=true,         %
         breaklines=true,            % 
         keywordstyle=\color{red},
    		frame=b,         
 %        keywordstyle=[1]\textbf,    % 
 %        keywordstyle=[2]\textbf,    %
 %        keywordstyle=[3]\textbf,    %
 %        keywordstyle=[4]\textbf,   \sqrt{\sqrt{}} %
         stringstyle=\color{white}\ttfamily, % 
         showspaces=false,           % 
         showtabs=false,             % 
         xleftmargin=17pt,
         framexleftmargin=17pt,
         framexrightmargin=5pt,
         framexbottommargin=4pt,
         %backgroundcolor=\color{lightgray},
         showstringspaces=false      % 
 }
 \lstloadlanguages{% Check Dokumentation for further languages ...
         %[Visual]Basic
         %Pascal
         %C
         %C++
         %XML
         HTML
         %,C#
         ,Java
         ,SQL
 }
    %\DeclareCaptionFont{blue}{\color{blue}} 

  %\captionsetup[lstlisting]{singlelinecheck=false, labelfont={blue}, textfont={blue}}
\usepackage{caption}
\DeclareCaptionFont{white}{\color{white}}
\DeclareCaptionFormat{listing}{\colorbox[cmyk]{0.43, 0.35, 0.35,0.01}{\parbox{\textwidth}{\hspace{15pt}#1#2#3}}}
\captionsetup[lstlisting]{format=listing,labelfont=white,textfont=white, singlelinecheck=false, margin=0pt, font={bf,footnotesize}}


\lstset{frame=single,breaklines=true,basicstyle=\footnotesize,numbers=right, numberstyle=\tiny\color{gray}}


\lstdefinestyle{listingsStyleJava}
{
    numberblanklines=false,
    language=make,
    tabsize=4,
    keywordstyle=\color{red},
    identifierstyle= %plain identifiers for make
}
% Aşağıdaki gibi kullanılıyor.
% \lstinputlisting[  firstline=1, caption={Deneme Java File},style=listingsStyleJava]{./src/makefile}



\begin{document}

\author{\authorName}
\title{\titleName}

\maketitle


\maketitle


Karkas code gen tarafından veri tabanından kod üretildiğinde,
TypeLibrary altında her veri tabanı tablosunun bir eşi olarak
bir c\# sınıfı, data model, üretilir. 
Bu üretilen c\# sınıflarına veritabanınan anlaşılabilecek olan
tüm scaffolding özellikleri eklenmektedir.
Ama Asp.Net MVC tarafından oluşturulan scaffolding kodunu
daha iyi yönetmek amacı ile üretilen
bu sınıflara ek annotationlar eklenebilir.

Bu annotationları bir kısmı System.ComponentModel.DataAnnotations namespace'ındedir.



Eklenebilecek bu annotationlar aşağıdaki listede \cite{Litwin2012} görülebilir.



\begin{itemize}

\item  Key

\item  StringLength

\item  Required

\item ScaffoldColumn

\item  MaxLength

\item ForeignKey

\item DatabaseGenerated

\item  NotMapped


\item  ConcurrencyCheck


\item  Timestamp

\item  ComplexType

\item  UIHint

\end{itemize}

Bunun haricinde Karkas ile kod geliştirilirken kullanılmayan
ama code first yaklaşımı ile Entity Framework ile kod
geliştirdiğinizde ihtiyacınız olabilecek olan
annotationlar aşağıdaki gibidir.


\begin{itemize}

\item  Column

\item Table

\item InverseProperty
\end{itemize}




\subsection{Key - Anahtar}
Key annotation'u işaretlediği kolonun bir primary key olduğunu belirtir.
Karkas tarafından otomatik olarak, veritabanı şema bilgisi kullanılarak eklenmektedir.


\subsection{StringLength - Katar Uzunluğu}
Bir string alanının ne kadar uzun olabileceğini belirten bir alandır.
Karkas tarafından otomatik olarak, veritabanı şema bilgisi kullanılarak eklenmektedir.
Bu özellik varchar, char gibi character bilgisi içeren kolonlara eklenmektedir.

\subsection{Required - Zorunlu Alan}
Bir özelliğin zorunlu olarak değer alıp almayacağını kontrol etmektedir.
Karkas tarafından otomatik olarak, veritabanı şema bilgisi kullanılarak eklenmektedir.
Zorunluluk veri tabanında not null verilen kolonlar için geçerlidir.

\subsection{ScaffoldColumn }
İlgili özelliğin scaffold işlemine girip girmeyeceğini kontrol etmektedir.
Karkas tarafından üretilen AsString property'lerinde kullanılır.
Bunun haricinde eğer bir kolonun scaffold edilmesini istemiyorsanız,
kod örneği \ref{code-ScaffoldColumn} gibi kullanabilirsiniz.




\begin{lstlisting}[label=code-ScaffoldColumn,caption=ScaffoldColumn,language=C]
		[ScaffoldColumn(false)]
		public string RegionIdAsString
\end{lstlisting}



		


\bibliographystyle{plain}


% \input{../latexAndBibliography/referanslar.bib}
\bibliography{latexAndBibliography/referanslar}






\end{document}