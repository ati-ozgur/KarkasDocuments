\documentclass[10pt,a4paper]{article}
\usepackage[utf8x]{inputenc}

\newcommand{\authorName}{Atilla \"Ozg\"ur}
\newcommand{\titleName}{Razor Kopya Kağıdı}


\usepackage{ucs}

\interdisplaylinepenalty=2500
\usepackage{algorithmic}

\usepackage{cite}



% *** MATH PACKAGES ***
%
\usepackage{amsfonts}
\usepackage{amsmath}
\usepackage{amssymb}



\usepackage{csquotes}
\usepackage{makeidx}
\usepackage{float}
\usepackage{longtable}
\usepackage[obeyDraft]{todonotes}
% \usepackage[disable]{todonotes}


\usepackage{cite}
\makeatletter
\@ifclassloaded{IEEEtran}{}{\usepackage{appendix}}
\@ifclassloaded{IEEEtran}{}{\usepackage{ltxcmds}}

\makeatother

% \usepackage{appendix}
\usepackage[final]{pdfpages}
\usepackage{import}

\usepackage{array}
% Frank Mittelbach's and David Carlisle's array.sty patches and improves
% the standard LaTeX2e array and tabular environments to provide better
% appearance and additional user controls. As the default LaTeX2e table
% generation code is lacking to the point of almost being broken with
% respect to the quality of the end results, all users are strongly
% advised to use an enhanced (at the very least that provided by array.sty)
% set of table tools. array.sty is already installed on most systems. The
% latest version and documentation can be obtained at:
% http://www.ctan.org/tex-archive/macros/latex/required/tools/


\usepackage{mdwmath}
\usepackage{mdwtab}
% Also highly recommended is Mark Wooding's extremely powerful MDW tools,
% especially mdwmath.sty and mdwtab.sty which are used to format equations
% and tables, respectively. The MDWtools set is already installed on most
% LaTeX systems. The lastest version and documentation is available at:
% http://www.ctan.org/tex-archive/macros/latex/contrib/mdwtools/


%% The lineno packages adds line numbers. Start line numbering with
%% \begin{linenumbers}, end it with \end{linenumbers}. Or switch it on
%% for the whole article with \linenumbers after \end{frontmatter}.
% \usepackage{lineno}


% \enquote{quote} şeklinde kullanılıyor.

 
 
 
\usepackage{caption}
\usepackage{subcaption}


\floatstyle{boxed} 
\restylefloat{figure}
\usepackage{multirow} 

\usepackage{hyperref}

% correct bad hyphenation here
\hyphenation{op-tical net-works semi-conduc-tor}

\usepackage{url}

\hypersetup{
    bookmarks=true,         % show bookmarks bar?
    unicode=true,          % non-Latin characters in Acrobat’s bookmarks
    pdftoolbar=true,        % show Acrobat’s toolbar?
    pdfmenubar=true,        % show Acrobat’s menu?
    pdffitwindow=false,     % window fit to page when opened
    pdfstartview={FitH},    % fits the width of the page to the window
    pdftitle={\titleName },    % title
    pdfauthor={\authorName },     % author
    pdfsubject={\titleName},   % subject of the document
    pdfcreator={\authorName},   % creator of the document
    colorlinks=false,       % false: boxed links; true: colored links
    linkcolor=red,          % color of internal links
    citecolor=green,        % color of links to bibliography
    filecolor=magenta,      % color of file links
    urlcolor=cyan           % color of external links
}



\usepackage{color}
\usepackage{courier}
\usepackage{xcolor}
\usepackage{listings}

\usepackage{caption}
\DeclareCaptionFont{white}{\color{white}}
\DeclareCaptionFormat{listing}{\colorbox{gray}{\parbox{\textwidth}{#1#2#3}}}
\captionsetup[lstlisting]{format=listing,labelfont=white,textfont=white}


\lstset{
         basicstyle=\footnotesize\ttfamily, % 
         %numbers=left,               % 
         numberstyle=\tiny,          % 
         %stepnumber=2,               % 
         numbersep=5pt,              % 
         tabsize=2,                  % 
         extendedchars=true,         %
         breaklines=true,            % 
         keywordstyle=\color{red},
    		frame=b,         
 %        keywordstyle=[1]\textbf,    % 
 %        keywordstyle=[2]\textbf,    %
 %        keywordstyle=[3]\textbf,    %
 %        keywordstyle=[4]\textbf,   \sqrt{\sqrt{}} %
         stringstyle=\color{white}\ttfamily, % 
         showspaces=false,           % 
         showtabs=false,             % 
         xleftmargin=17pt,
         framexleftmargin=17pt,
         framexrightmargin=5pt,
         framexbottommargin=4pt,
         %backgroundcolor=\color{lightgray},
         showstringspaces=false      % 
 }
 \lstloadlanguages{% Check Dokumentation for further languages ...
         %[Visual]Basic
         %Pascal
         %C
         %C++
         %XML
         HTML
         %,C#
         ,Java
         ,SQL
 }
    %\DeclareCaptionFont{blue}{\color{blue}} 

  %\captionsetup[lstlisting]{singlelinecheck=false, labelfont={blue}, textfont={blue}}
\usepackage{caption}
\DeclareCaptionFont{white}{\color{white}}
\DeclareCaptionFormat{listing}{\colorbox[cmyk]{0.43, 0.35, 0.35,0.01}{\parbox{\textwidth}{\hspace{15pt}#1#2#3}}}
\captionsetup[lstlisting]{format=listing,labelfont=white,textfont=white, singlelinecheck=false, margin=0pt, font={bf,footnotesize}}


\lstset{frame=single,breaklines=true,basicstyle=\footnotesize,numbers=right, numberstyle=\tiny\color{gray}}


\lstdefinestyle{listingsStyleJava}
{
    numberblanklines=false,
    language=make,
    tabsize=4,
    keywordstyle=\color{red},
    identifierstyle= %plain identifiers for make
}
% Aşağıdaki gibi kullanılıyor.
% \lstinputlisting[  firstline=1, caption={Deneme Java File},style=listingsStyleJava]{./src/makefile}


\renewcommand\lstlistingname{Kod Listesi}
\renewcommand\lstlistlistingname{Kod Listeleri}



\begin{document}

\author{\authorName}
\title{\titleName}



\maketitle



Referans listesinde görebileceğiniz \cite{Haack2011C},\cite{Haley2010Developer} kaynaklardan
yararlanılarak yazılmıştır.


Razor yapı olarak oldukça basit bir mantıkla çalışır.
Normal html kodu yazılır gibi html sayfanızı yazarsınız,
C\# kodu yazmak için @ işaretini kullanırsınız.
Html ve C\# kodunun iç içe rahatça kullanılması için tasarlanmış bir dildir.

Aşağıda kullanıbileceğiniz yapılara örnekler verilmektedir.

\section{Razor Yapı Örnekleri}


\subsection{Multiline code block - Çok Satırlı kod örneği.}

Çok satırlı kod örneklerinin cshtml dosyasının başında yazılmaları bu sayede normal
html içeriği ile çok karışmaması daha uygundur.

\lstinputlisting[label=code-MultiLineRazor,caption=Çok Satırlı Kod Örneği]{CodeExamples/Razor/MultiLineRazor.cshtml}






\subsection{Html İçinde Kod Örnekleri}


\begin{lstlisting}[label=code-ExampleOfInlineExpression1,caption=html içinde kod örneği 1]
<script src="@Url.Content("~/Scripts/MSAjaxHistoryBundle.js")" type="text/javascript"></script>
\end{lstlisting}

Asp.Net MVC4 ile ~ işaret src ve img taglarında otomatik olarak parse edilmektedir. 
Bundan dolayı kod listesi \ref{code-ExampleOfInlineExpression1} aşağıdaki kod listesi \ref{code-ExampleOfInlineExpression2}  gibi yazılabilir.

\begin{lstlisting}[label=code-ExampleOfInlineExpression2,caption=html içinde kod örneği 2]
<script src="~/Scripts/MSAjaxHistoryBundle.js" type="text/javascript"></script>
\end{lstlisting}


\begin{lstlisting}[label=code-ExampleOfInlineExpression3,caption=html içinde kod örneği 3]
@model.Adi
\end{lstlisting}

Normal kod örnekleri html encode edilmektedir.
Eğer html encode edilmeden çıktı alınması isteniyorsa @Html.Raw kullanılmalıdır.
Bu tür bir ihtiyaç genellikle girdi sırasında html veya xml taglarına izin veriliyorsa ihtiyaçtır.
Mesela Rich Editor (Zengin Editor) ile html yorumlarına izin verildiğini düşünelim.
Bu durumda html içeriğin kod listesi \ref{code-ExampleOfInlineExpressionHtmlRaw1}
gibi verilmesi gerekmektedir

\begin{lstlisting}[label=code-ExampleOfInlineExpressionHtmlRaw1,caption=html içinde kod örneği 1 html encode yapılmadan]
@Html.Raw(@model.Yorum)
\end{lstlisting}

Encode edilen ve edilmeyen string farkını daha iyi anlamak  için kod listesi \ref{code-ExampleOfInlineExpressionHtmlRaw2} çalıştırıp sonucuna bakınız.

\begin{lstlisting}[label=code-ExampleOfInlineExpressionHtmlRaw2,caption=html içinde kod örneği 2 html encode yapılmadan ]
@{
 String cikti = "<span>hi</span>";
}
@Html.Raw(cikti)
@cikti


\end{lstlisting}


\subsection{Yerel Değişken oluşturma ve kullanma}


\begin{lstlisting}[label=code-LocalVariableUse,caption=Yerel Değişken oluşturma ve kullanma]
@{
	DateTime bugun = DateTime.Now.ToShortDateString();
}

Bugun  @bugun

\end{lstlisting}


\subsection{Durum Kontrol Örnek - if}



\begin{lstlisting}[label=code-ConditionalVariableExample1,caption=Durum Kontrol Örnek - if]


<strong>Bugun</strong>
@if(DateTime.Now.DayOfWeek == DayOfWeek.Monday){
    <p>Pazartesi, Pazartesi Sendromu cabuk gesse bari </p>
}





\end{lstlisting}



\subsection{Text \& Markup -- Yazı ve Markup}

\begin{lstlisting}[label=code-TextAndMarkup,caption=Yazı ve Markup]
@foreach(Musteri m in musteriListesi) {
  <span>@m.Adi</span> 
  <span>@m.Soyadi</span> 
}
\end{lstlisting}


\subsection{Kod ve Sade Yazı}


\begin{lstlisting}[label=code-TextAndMarkup1,caption=Yazı ve Markup 1 text tag]
if(yanlisMi) {
 <text>Deger Yanlis</text>
}
\end{lstlisting}

\begin{lstlisting}[label=code-TextAndMarkup2,caption=Yazı ve Markup 2]
@if(DateTime.Now.DayOfWeek == DayOfWeek.Sunday){
    <p>Pazar haftanin en guzel gunu.</p>
}\end{lstlisting}


iki örnek arasındaki fark şudur. 
Razor html ile c\# iç içe geçmesi için tasarlanmıştır.
Bundan dolayı kod listesi \ref{code-TextAndMarkup2} içindeki p tagı ile html'e geçiş yapılmaktadır.
Ama sade yazı yazmaya çalıştığımızda razor , normal yazımızı c\# kodu gibi algılamaktadır.
Bundan dolayı text tagını kullanarak , yazı yazmak istediğimizi belirtiyoruz.
Eğer html taglarından herhangi birini kullanırsak, text tagına ihtiyacımız kalmayacaktır.


text tagına alternatif olarak @: ile sade yazı yazılabilir, bakınız kod listesi \ref{code-TextAndMarkup3}.
Ben text tagının daha temiz ve anlaşılabilir olduğunu düşünüyorum.

\begin{lstlisting}[label=code-TextAndMarkup3,caption=Yazı ve Markup 3 @:]
if(yanlisMi) {
  @:Deger Yanlis
}
\end{lstlisting}





\subsection{Kod , Sade Yazı, Kod}

\begin{lstlisting}[label=code-TextAndMarkup,caption=Yazı ve Markup]
if(yanlisMi) {
  @:Bu degisken Yanlistir. @yanlisMi
}
@ tekrar kod donus

<text>Bu tag ile yazi yaziliyor. Su an zaman @@zaman </text>
\end{lstlisting}


\subsection{Comments - Sunucu Tarafında Yorum}

\begin{lstlisting}[label=code-Comment,caption=Comment - Yorum]
@* Bu satirlar
 browser tarafinda  
 gorulmeyecektir. *@
\end{lstlisting}


\subsection{Layout ve \_ViewStart Dosyaları}
cshtml viewlarının master sayfalarının veya razor view deyimi ile layoutları
view başındaki aşağıdaki gibi bir kod ile set edilir.

\begin{lstlisting}[label=code-Layout,caption=Layout - Düzenleme]
@{
	LayoutPage = "~/Views/Shared/_Layout.cshtml";
}
\end{lstlisting}




\subsection{Global Namespace}
Eğer viewların hepsinde kullanmak istediğimiz namespace'ler varsa bunları tüm view'lar için View dizini altındaki web.config
içine ekleyebiliriz.

\begin{lstlisting}[label=code-GlobalNamespace,caption=Global Namespace Ekleme]
<pages>
   <namespaces>
      <add namespace="System.Web.Mvc" />
      <add namespace="Karkas.WebExtensions" />
\end{lstlisting}


\subsection{View'a Namespace Ekleme}
\begin{lstlisting}[label=code-ViewNamespace,caption=View Namespace Ekleme]
@using Karkas.Deneme.Helpers
\end{lstlisting}



\subsection{Content Placeholder - İçerik yertutucu }

Bu yapıda layout içinde boş bırakılan yerler, view tarafından doldurulur.
Örneğin layout içinde ana içeriğin geleceği yer @RenderSection veya @RenderBody
ile belirtilir.
View içerisinde view çıktısının bu boş bırakılan hangi yere geleceği yazılır.
Yer tutucu kodunun çalışması için hem layout içerisinde hemde view içerisinde tanımı olmalıdır.
Yer tutucu Layout içerisinde optional anahtar kelimesi ile opsiyonel olarak tanımlanabilir.



\begin{lstlisting}[label=code-ContentPlaceHolderLayout,caption=İçerik yertutucu - Layout ]

<body>


.....
       @RenderSection("ScriptYerTutucu", optional:true)

</body>

\end{lstlisting}


\begin{lstlisting}[label=code-ContentPlaceHolderView,caption=İçerik yertutucu - View ]

@section ScriptYerTutucu {

	// Buraya o view icin gerekli javascript kodu yazilabilir.
	
}

\end{lstlisting}


\subsection{Eposta adresi}

Razor normal yazımlarda eposta adresini anlayabiliyor. 
Ayrıca @ işaretinden kaçmamız gerekmiyor.
\begin{lstlisting}[label=code-EmailExample,caption=EPosta Örnek ]
EPosta Adresi
ati.ozgur@ornek.com
\end{lstlisting}

\subsection{Explicit Expression - Belirgin İfade }
Bazı durumlarda, kod listesi \ref{code-ExplicitExpression} , 
@ işareti ile çağrılacak değişkenin parentez içinde belirtilmesi gerekir.


\begin{lstlisting}[label=code-ExplicitExpression,caption=Belirgin İfade  ]
<span>ISBN@(isbnNumber)</span>
\end{lstlisting}


\subsection{Escaping the @ sign - @ işaretinden kaçmak }


\begin{lstlisting}[label=code-EscapingAtSign ,caption= @ işaretinden kaçmak  ]
<span>Razor kullanirken @@adi ile adi degiskeninin degerini gosterirsiniz. </span>

\end{lstlisting}


\subsection{Generik Method Çağırma }

\begin{lstlisting}[label=code-CallingGenericMethod ,caption=Generik Method Çağırma  ]
@(GenericSinif.GenericMethod<AType>())

\end{lstlisting}



\bibliographystyle{plain}


% \input{../latexAndBibliography/referanslar.bib}
\bibliography{latexAndBibliography/referanslar}



\end{document}